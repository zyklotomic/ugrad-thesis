\chapter{Introduction}

Currently, a popular and well-tested approach to efficient asynchronous I/O is through the
use of polling mechanisms. Depending on the operating system, system calls such as \texttt{epoll}, \texttt{kqueue}, or \texttt{select},
can be used to register a file descriptor for interest in potentially blocking I/O,
as well as receive notification that the file descriptor is ready and guaranteed to not
block on the desired I/O operation. This design has also been the basis of concurrent Haskell
programs.
The design was refined upon, into what is now the MIO \texttt{EventManager}
\cite{mio, scalableIO},
where now multiple cores could each manage their own pending I/O notifications, and schedule their respective threads.

More recently, Linux introduced \texttt{io\_uring}, a new asynchronous system call facility that uses a
completion-based paradigm, fundamentally different from the readiness-based paradigm of polling mechanisms.
In \texttt{io\_uring}, instead of waiting for the kernel to notify user threads of readiness events as in epoll,
a task carrying a request to perform a potentially blocking I/O operation is sent to the kernel,
and when the operation is complete, the kernel responds with the result. Theoretically,
the design is expected to be more efficient
due to fewer syscalls required. However, in practice, the possible
performance gains have come into question.

Hence, our motivation for writing \texttt{URingManager}: firstly trying to gauge the performance of \texttt{io\_uring},
especially with Haskell’s lightweight green threading-based runtime system, and secondly,
exploring this new paradigm of asynchronous programming.
