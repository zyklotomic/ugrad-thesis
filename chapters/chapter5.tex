\chapter{Optimizations}
\section{Kernel Polling Mode} 
Instead of submitting the \texttt{io\_uring\_enter} system call to notify the kernel
of new entries, we can use a kernel thread to continuously check the submission
queue for new entries. To enable this is as simple as enabling the
\texttt{IORING\_SETUP\_SQPOLL} on setup.
Again, this is in the \texttt{io\_uring} theme of reducing syscalls.

\section{Multi-shot SQEs}

In general, each SQE being submitted corresponds to a single CQE received.
However, in newer kernel versions, multi-shot SQEs have been introduced.
As opposed to the existing one-shot SQEs, multi-shot SQEs can return multiple CQEs when applicable.
For instance, the \texttt{accept()} system call is used to assign the next new connection to a new socket file descriptor.
In the case of both the regular synchronous \texttt{accept()} and
the \texttt{io\_uring} \texttt{IORING\_OP\_ACCEPT} asynchronous request,
upon exiting and receiving each new connection, another accept syscall has to be made again to accept the next connection.
With the multishot variant of \texttt{IORING\_OP\_ACCEPT},
the kernel will continuously queue up new CQEs containing connection results without
the user process having to rearm the submission queue with another accept request.
This is more efficient because in the non-kernel polling case, this saves an
\texttt{io\_uring\_enter} syscall, and in either case, saves the kernel from having to process an additional SQE.

To implement support for multi-shot SQEs, we modify the requests table
inside the \texttt{URingManager} to keep track of whether a request is multi-shot or one-shot. 
Additionally, instead of deleting the entry upon receiving a CQE corresponding to a
multi-shot SQE, we simply keep it in the requests table. Last but not least,
for multi-shot SQE submissions, rather
than a callback that does \texttt{putMVar}, we store a callback that spawns a new Haskell
thread bounded to the same Capability (via \texttt{forkOn} because the direct file table
is local to that specific Capability) that handles the results instead.

% STRETCH GOAL
% Figure X shows the pertinent code changes.

\section{Fixed File Descriptors}

At the kernel level,  in multi-threaded contexts, sharing a file descriptor table has a non-trivial overhead.
A recent new development in \texttt{io\_uring} allows the usage of thread-local file descriptors and file descriptor tables, in what is referred to as fixed file descriptors.
Fixed file descriptors have the same semantics as regular file descriptors, except they are managed by a user-allocated local file descriptor table.
Therefore, these local fixed file descriptors are only valid in the same thread.
These fixed file descriptors are operated on what is referred to as direct variants of existing syscalls/wrapper functions.
For example, in liburing, there is both an \texttt{io\_uring\_prep\_accept()} and an \texttt{io\_uring\_prep\_accept\_direct()} wrapper function.

Furthermore, it is also possible to pre-assign fixed file descriptors
to syscall requests, such as \texttt{open()}-ing a file at a specific fixed file descriptor.
However, we do not take advantage of this pre-assigning ability in our \texttt{URingManager} design.

In order to accommodate fixed file descriptors in our \texttt{URingManager}, we most significantly had to change
the way we scheduled our Haskell lightweight threads.
In our original design, it was possible for a Haskell thread to submit an SQE to different
\texttt{URingManager}s and therefore \texttt{io\_uring} submission queues between different requests.
This is possible because the \texttt{URingManager} interface only allows for submitting the SQE request
to the \texttt{URingManager} corresponding to the Capability that is currently executing the submitting Haskell thread,
and Haskell threads are not necessarily bound to a specific Capability.
The solution is to make sure that Haskell threads using fixed file descriptors in their
logic need to be spawned with \texttt{forkOn} in order to be tied to a Capability.


% TODO: STRETCH GOAL
% Compare Figure Y to Figure Z to see the corresponding differences in code in order to
% support fixed file descriptors and multi-shot SQEs.
