\chapter{Benchmarking}
\section{Web Server Benchmark Program}
We slightly modify the simple HTTP web server program
used to benchmark MIO \cite{mio} for own benchmark.
We use our version of the \texttt{network} library that supports our URingManager. The web server, listening on a specified address, receives HTTP requests and then responds with a fixed HTTP response regardless of received content.

When benchmarking, we test two versions of the server: one without URingManager (using default \texttt{epoll EventManager}) and one with \texttt{URingManager}. We initially wished to benchmark a modified version of our \texttt{URingManager} with all the mentioned optimizations in the previous section, however, our implementation unfortunately would frequently hang. This made it impossible to reliably test. 
The main parts of the server are shown in Figure \ref{fig:WebServer}

\begin{figure}[ht]
    \centering
	\lstinputlisting[language=Haskell]{figures/code/WebServer.hs}
	\caption[Benchmarked Web Server Source]{Main parts of our web server}
	\label{fig:WebServer}
\end{figure} 

The sources are all included in the monorepo \url{https://github.com/zyklotomic/uring-manager}.

\section{Setup} 
We used Apache Bench (\texttt{ab}) to measure the time taken for 10,000  default HTTP requests of size 151 bytes each with a variable number of concurrent connections to be processed by our web server.

Hardware-wise, an AMD Ryzen 5 2600 Six-Core Processor with 16 GB of RAM was used. The Linux kernel version used was 6.4.3 on the NixOS distribution.

Both the Apache Bench and the web-server program were running on the same CPU. In order to minimize variance caused by scheduling, the Linux \texttt{cpuset} mechanism was used to reserve 8 cores for the web-server program only, and 1 core for the single-threaded Apache Bench. Kernel threads were allowed to run on the aforementioned reserved cores.

